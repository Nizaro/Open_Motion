
\documentclass[twocolumn]{bmcart}


%%%%%%%%%%%%
% Packages
%%%%%%%%%%%%

\usepackage[utf8]{inputenc} %unicode support

\usepackage{graphics} % for pdf, bitmapped graphics files
\usepackage{epsfig} % for postscript graphics files
\usepackage{amsmath} % assumes amsmath package installed
\usepackage{amssymb}  % assumes amsmath package installed
\usepackage{subcaption}



%\def\includegraphic{}
%\def\includegraphics{}



%%%%%%%%%%%%
% Local definition
%%%%%%%%%%%%

\startlocaldefs

% quaternion style
\def\mathbi#1{\textbf{\em #1}}

% roman letter
\makeatletter
\newcommand*{\rom}[1]{\expandafter\@slowromancap\romannumeral #1@}
\makeatother


\endlocaldefs


%%%%%%%%%%%%
% Begin Document
%%%%%%%%%%%%
\begin{document}


%%%%%%%%%%%%
% Title
%%%%%%%%%%%%

\begin{frontmatter}

\begin{fmbox}
\dochead{Research}


\author[
   addressref={aff1},                   % id's of addresses, e.g. {aff1,aff2}
   corref={aff1},     
   noteref={n1},                        % id's of article notes, if any
   email={nizar.ouarti@ipal.cnrs.fr}   % email address
]{\inits{NO}\fnm{Nizar} \snm{Ouarti}}
\author[
   addressref={aff1},   corref={aff1},  
   email={braud.thomas@ipal.cnrs.fr}
]{\inits{TB}\fnm{Thomas} \snm{Braud}}
\author[
   addressref={aff1},   corref={aff1},  
   email={vivien.billaud@ipal.cnrs.fr}
]{\inits{VB}\fnm{Vivien} \snm{Billaud}}



\address[id=aff1]{%                           % unique id
  \orgname{IPAL (Sorbonne UPMC, CNRS, Astar, NUS, UJF, IMT)}, % university, etc
  \city{Singapore}
  }


\end{fmbox}% comment this for two column layout

 \title{OpenMotion: An open source library for attitude estimation.} 

\maketitle


%%%%%%%%%%%%
% Abstract + Introduction
%%%%%%%%%%%%


\begin{abstractbox}

\begin{abstract} % abstract

The abstract goes here
\end{abstract}

\begin{keyword}
\kwd{IMU}
\kwd{Sensor Fusion}
\kwd{Sensor Calibration}
\end{keyword}

\end{abstractbox}


\end{frontmatter}

\section{Introduction}


%context + motivation

Since the last decades, it is easily observable that smartphones market has grown drastically. Indeed, this devices propose convenient functionalities for household. On the other hands, the different kind of sensor available and easily accessible make smartphones an interesting tool for research.  Nowadays, some applications where smartphone can be used attract the attention of researchers such as Indoor Navigation\cite{huang2009survey}. Severals solution for this has been proposed with different approaches like using only computer vision technics \cite{desouza2002vision,bonin2008visual} or wireless networks\cite{gu2009survey}. Another approaches is to sense the motion by using some inertial sensors. Among its sensor and according to the smartphone, an Inertial Measure Unit (IMU) is vacant and used when coupling with a magnetometer as an Attitude and Heading Reference System (AHRS). However, an AHRS sensor may be not robust or accurate in certain situation due to unpredictable event or sensor deficiencies. Moreover, for Indoor Localization or Navigation applications, it is crucial to have an algorithm that propose an optimal estimation of the attitude in term of accuracy and computation time requirement. This is the reason why researchers are interested in sensor fusion approaches because it overcomes some problems such as sensor drifting. 

Since 1960, severals data fusion has been developed after a seminal framework on data fusion has been initiated by Kalman and  Bucy \cite{kalman_new_1960}. In this paper, the most common family of data fusion algorithm is presented briefly. However, implementing this kind of algorithm require time and resources. Furthermore, the available software of data fusion remains general and may be not adaptable for the inertial case. For this reason, we propose an open source and free library called \texttt{OpenMotion} that contains different family of sensor fusion algorithms designed to fit to the inertial case. For this paper, the different functionalities proposed by the library are presented. In a simulated environment, a performance study of all methods has been processed and the values are shown. To quantify the performances, we determined 2 kind of scores,  $s_1$ representing the accuracy of the method and $s_2$ that takes into account the accuracy and the computation time. The $s_2$ give better visibility of the methods' effectiveness for usage in embedded devices, which is interesting for smartphone application. Furthermore, working only in a simulated environment is not enough to have a good visibility. Moreover, in our context, it is very difficult to obtain a reliable ground truth. For this study, we designed the ground truth  based on computer vision methods. 

%%%%%%%%%%%%
% General Presentation of the library
%%%%%%%%%%%%


\section{Global Presentation of the Library}

\texttt{OpenMotion} is a open source library mainly aimed for real-time attitude estimation. Our approach is based on data fusion technics using only the output of a IMU (composed by a 3D gyroscope,  3D accelerometer and a 3D magnetometer). Among several sensor fusion algorithms proposed by the library, the user has the possibilities to choose the one that fits the best to his expectation according to its performance. 

However, we designed the library also as an academic tool. For this reason, we have chosen to implement for this library the most common family of methods. In previous works \cite{braudcomparison}, we proposed a comparative framework that allow us to observe the performance of this different methods in a quantitative manner. Thus, the user has in his disposition the code and and the performance study. This will help him to have a better understanding of the algorithms' mechanism and behavior. On the other hand, the user can use the library as a developing tool to design and create new sensor fusion algorithm. Moreover, he can compare his work directly with the available methods. For this paper,  2 references frames are used as follow:


\vspace{0.1cm}

\begin{itemize}
\item The North East Down (NED) frame $\{a\}$ system has its origin fixed at the (moving) object center of gravity. The $z$-axis points upward perpendicularly to the tangent plane of the ellipsoid, and the $x$-axis points towards true north (and not the magnetic north). The $y$-axis point towards east.

\vspace{0.1cm}

\item The object-fixed reference frame $\{c\}$ corresponding to the IMU device. It is a moving and rotating coordinate frame. 
\end{itemize}

\vspace{0.1cm}

For simplicity reasons, the library admit that all the components of the IMU belong to the same reference frame $\{c\}$ (ref figure \ref{Schema_situation}). It is a cyclopean approximation \cite{ouarti2008multimodal}. Moreover, the user has to interface the sends of data from the device to the library because the library does not take care of it. \texttt{OpenMotion} provide an estimation of the orientation of $\{c\}$ relative to $\{a\}$. The choice of the output type (quaternion, rotation matrix, euler angle) is also according to the user needs. 

\begin{figure}
\centering
\includegraphics[scale=0.65]{images/Schema_situation.png}
\caption{Definition of the scene with 2 frame:  a fixed frame NED (North East Down) noted $\{a\}$ and a moving frame object noted $\{c\}$. \texttt{OpenMotion} will provide an estimate of the attitude of $\{c\}$ relative to $\{a\}$. }
\label{Schema_situation}
\end{figure}

We are also working on the establishment of a dynamic sensor calibration process on all component of the IMU. We insist on the dynamic aspect. Indeed, firstly, it allows  the library to be adaptable to any kind of IMU, but also to improve the performance without requesting some effort from the user.



%%%%%%%%%%%%
% Presentation of the calibration process
%%%%%%%%%%%%

\section{Calibration and initialization}

\subsection{Problematique de la calibration}

	Les systemes embarque comme les smartphones disposent pour la plupart de capteurs servant au systeme ou a l'utilisateur pour se renseigner sur des grandeurs utiles quant a l'appareil lui-meme ou bien a son environnement. Ces capteurs sont donc varies mais on peut notamment distinguer les capteurs constituant une centrale inertielle ou IMU pour Inertial Measurement Unit que sont les accelerometres et les gyroscopes 3 axes. Ces capteurs sont de plus en plus associes avec un magnetometre. Ainsi, ces capteurs peuvent constituer un AHRS pour Attitude and Heading Reference System ayant pour but de reperer l'appareil dans l'espace 3D. L'accelerometre sert ici a mesurer l'acceleration lineaire, le gyroscope la vitesse angulaire et le magnetometre le champ magnetique terrestre local et sert dans la plupart des cas de compas. Dans le cas d'un smartphone, ces capteurs sont low-cost et de type MEMS pour Micro-Electro-Mechanical Systems. Des lors, le signal utile etant faible, l'erreur induite peut-etre du meme ordre de grandeur et donc conduire a des imprecisions importantes, notamment dans le cas des AHRS qui font l'objet d'une fusion de donnees. Il convient donc d'eliminer ou tout du moins de reduire ces erreurs via une calibration initiale de l'ensemble des capteurs utilises.

\subsection{Types et modele d'erreur}

	Tous les types de capteurs presentent des biais, facteurs d'echelle (scale factors), des erreurs d'inter-couplage (misalignements et cross axis sensitivities) et dans une certaine mesure du bruit aleatoire. Des erreurs d'ordre superieur, des erreurs d'inter-couplage d'acceleration angulaire ou encore des erreurs specifiques comme les softs et irons distorsions pour le magnetometre peuvent egalement se produire.
Pour l'ensemble des capteurs, un modele rendant compte des erreurs lors des differentes mesures peut-etre propose :

\begin{equation}
	X_i = MS \cdot \left( X_{i,raw} - \textbf{b} \right) + \textbf{n}_i
\end{equation}


ou $X_i$ est la mesure du capteur comprenant les erreurs, $X_{i, raw}$ la mesure reelle du capteur,  $b$ le biais,  $S$ la matrice rendant compte des scales factor errors, $M$ la matrice des misalignement entre les axes du capteur et $\textbf{n}_i$ le bruit aleatoire avec $i \in \{x,y,z\}$.
Les differents matrices sont alors \\

\begin{center}
$S = \begin{pmatrix}  	s_{xx} & s_{xy} & s_{xz} \\ s_{yx} & s_{yy} & s_{yz} \\s_{zx} & s_{zy} & s_{zz}  \end{pmatrix}$ 	
 et
 $M = \begin{pmatrix}  1 & -\alpha_{yz} & \alpha_{zy} \\  0 & 1 & -\alpha_{zx} \\  0 & 0 & 1 \end{pmatrix} $
\end{center} 

\vspace{0.2cm}

ou $\alpha_{ij}$ avec $i,j \in \{x,y,z\} $ est l'angle de misalignement entre l'axe $i$ du repere du capteur avec celui de la plateforme selon l'axe $j$ de la plateforme. Enfin le biais est represente par un vecteur $\textbf {b} \in \mathbb{R}^3$


\subsection{Methodes adoptees}

Une premiere calibration d'usine est effectuee mais s'avere souvent imprecise ou obsolete du fait de l'usure des capteurs. Des lors, une nouvelle calibration s'avere necessaire.

La calibration des capteurs se fera dans un premier temps independamment les unes des autres et de maniere statique emph{i. e.} a l'initialisation. Il est a noter que nous ne prendrons pas en compte ici l'influence de la temperature puisqu'il est deja traite dans la calibration d'usine.

\subsubsection{Accelerometre}

Freescale propose une methode de calibration lineaire de l'accelerometre 3 axes. Elle repose sur le fait que la somme des amplitudes des composantes normalisees du champ gravitationnel terrestre sur les 3 axes du capteur doit equationetre egale a l'amplitude totale de celui-ci quand le capteur est dans un etat quasi statique. Elle consiste a placer l'appareil dans 8 positions donnees pour chaque axe puis d'utiliser une methode d'optimisation de moindres carres pour determiner les 12 parametres de calibration que sont les scale factors et le biais.

Cette methode ne permet toutefois pas de determiner les misalignements. 

\subsubsection{Gyroscope}

Zhang  propose une methode de calibration du gyroscope\cite{Gyroscope}. Elle repose sur le fait que la somme des mesures selon les 3 axes du gyroscope doit equationetre egale a la vitesse angulaire en entree. 

2 etapes sont alors necessaires a la calibration du gyroscope. La premiere consiste en des rotations horaires et anti-horaires permettant d'eliminer la vitesse angulaire de rotation de la terre et le biais du gyroscope en prenant le carre de la norme de la difference des integrales des mesures dans les 2 sens puis en utilisant une methode de moindre carre. Les scales factors et les angles de misalignements sont ainsi determines. Ensuite, les angles de misalignements et les scales factors sont utilises pour estimer le biais via la difference des carres de l'equation de la somme des vitesses angulaires pour 2 positions differentes.

\subsubsection{Magnetometre}

	Vasconcelos et al. \cite{MLE} proposent un algorithme dit de maximum likelihood estimation (MLE) qui utilise l'estimation issue de "Two-Step" \cite{Two-Step}, $i. e.$ une methode d'estimation de moindre carre suivie d'un filtrage etendu de Kalman (EKF), comme approximation initiale et qui traite le bruit de mesure comme une distribution normale.

	Le probleme de minimisation  la fonction  de vraisemblance (likelihood) qui est quadratique revient alors a determiner les points de l'ellipsoide  qui correspondent au mieux aux mesures du capteur.

	Ce probleme est resolu via la methode du gradient et une methode de Newton descendante pour des espaces euclidiens \cite{Newton} et la regle de Armijo pour la determination de la taille du pas. Le gradient et la matrice hessienne de la fonction de log-likelihood sont eux calcules analytiquement. 

\subsubsection{Calibration on-board}

	Une calibration on-board peut egalement s'averer necessaire du fait d'un besoin de precision accrue ou de l'obsolescence de la calibration d'usine due a l'usure des capteurs.



%%%%%%%%%%%%
% State of the art
%%%%%%%%%%%%


\section{State of the art}

\subsection{Quaternion representation}

A quaternion is a four-dimensional vector, defined as 

\begin{center}
$ \mathbi{q} = \begin{pmatrix} \textbf{q} \\ q_w \end{pmatrix} $ with $ \textbf{q} =  ( q_x, q_y, q_z)^T $. 
\end{center}

For attitude representation, the quaternion must satisfies a single constraint given by  $|| \mathbi{q}  || = 1$.

Then, the attitude matrix $R \in SO(3) $ related to the quaternion $\mathbi{q}$ is given by :
\vspace{-0.15cm}
\begin{equation}
R(\mathbi{q}) = (q_w^2 - |\textbf{q}|^2)I + 2 \textbf{q}\textbf{q}^T - 2q_w[\textbf{q} \times]
\label{quat_to_rot}
\end{equation}

Where $SO(3)$ is the special orthogonal group defined by:
\vspace{-0.15cm}
$$SO(3) = \{ R \in \mathbb{R}^{ 3 \times 3} | RR^T = R^TR = I, det(R) = 1 \}$$

And $[\textbf{v} \times] \in \mathbb{R}^{3\times 3}$ is the skew-symetric matrix of the vector $\textbf{v}$ such as.

\begin{equation}
[\textbf{v} \times] = \begin{pmatrix} 0 & v_z & -v_y \\ -v_z & 0 & v_x \\ v_y & -v_x & 0 \end{pmatrix} 
\label{skewsymmat}
\end{equation}


\subsection{Most Common Fusion Methods}


\subsubsection{Kalman Filter}

Introduced in 1960 by Kalman \cite{kalman_new_1960}, the Kalman filter is an algorithm that uses a series of measurements observed over time, containing noise, in order to estimate unknown variables (state vector). Based on a model it is possible to obtain a more robust estimation of the state vector. Consider the following linear system, described by the difference equation and the observation model with additive noise:


\begin{equation}
\left\{ \begin{array}{l}
\textbf{x}_{k+1} = F_{k}\textbf{x}_{k}+B_{k}\textbf{u}_{k}+G_{k}\textbf{v}_{k}\\
\textbf{z}_{k} = H_{k}\textbf{x}_{k}+D_{k}\textbf{w}_{k}
\end{array} \right.
\label{linear_system}
\end{equation}

with,

\begin{center}
\begin{tabular}{rl}
$\textbf{x}_{k} $ &State vector at time $k$\\
$\textbf{z}_{k} $ &Measurement vector at time $k$\\
$\textbf{u}_{k} $ & Input vector (e.i.: from sensor) at time $k$ \\
$\textbf{v}_{k} $ & State noise vector disturbing the system\\
$\textbf{w}_{k} $ & Observation noise disturbing the measurement\\
\end{tabular}
\end{center}

\vspace{0.2cm}


A brief presentation of the algorithm is provided here,  see \cite{terejanu2013discrete} for more details. The algorithm has 3 important steps :\\

\begin{itemize}
\item The first step is the \textbf{prediction} of state $\textbf{x}^-_{k+1}$ at time $k+1$ and covariance $P^-_{k+1}$ is given by :

\begin{equation}
\left\{ \begin{array}{cl}
\textbf{x}^-_{k+1} = & F_{k}\textbf{x}^+_{k} + B_{k}\textbf{u}_{k} \\
P^-_{k+1} = & F_{k}P^+_{k}F^T_{k}+G_{k}Q_{k}G^T_{k}
\end{array}
\right.
\end{equation}

\item The \textbf{innovation} is the difference between predicted measurement and real measurement. It measure up the deviation used by the Kalman gain.
\begin{equation}
\left\{ \begin{array}{cl}
\boldsymbol\nu_{k+1} = & \textbf{z}_{k+1}-H_{k+1}\textbf{x}^-_{k+1} \\
\mathcal{S} = & R_{k+1} +H_{k+1}P^-_{k+1}H^T_{k+1}
\end{array}
\right.
\end{equation}

\item And finally, An \textbf{update} is processed to correct the estimate state $\textbf{x}^+_{k+1} $ and his covariance $P^+_{k+1}$ with the innovation.

\begin{equation}
\left\{ \begin{array}{cl}
\textbf{x}^+_{k+1} = & \textbf{x}^-_{k+1} + \mathcal{K}_{k+1}\boldsymbol\nu_{k+1} \\
P^+_{k+1} = &P^-_{k+1} - \mathcal{K}_{k+1}S_{k+1}\mathcal{K}^T_{k+1}
\end{array}
\right.
\end{equation}

\end{itemize}

Where $\mathcal{K}_{k+1}$ is the Kalman gain  representing relative importance of innovation  $\nu_{k+1}$. 

\begin{equation}
\mathcal{K}_{k+1}=P^-_{k+1}H^T_{k+1}\mathcal{S}^{-1}
\end{equation}

The Kalman filter provides an optimal estimation for linear systems. Nevertheless, most of studied dynamical systems are nonlinear (attitude estimation, pendula, ... ). Thus, extended Kalman Filter approach  was proposed for nonlinear systems \cite{larson1967application, larson1967precomputation,terejanu2008extended}. 

Here, we have chosen to use quaternions for the attitude representation because of its singular-free specificity. But special care must be taken when designing a quaternion based extended Kalman filter. Because of the constraint on the quaternions, the covariance matrix $P$ can be singular \cite{vik2009integrated} which is problematic for the numerical stability. There exist several methods to deal with this issue. Multiplicative Extended Kalman Filter (MEKF)\cite{markley2003attitude} and the Additive Extended Kalman Filter (AEKF) \cite{bar1991quaternion} are designed to tackle this problem.The difference between AEKF and MEKF is discussed in \cite{markley_multiplicative_2004}. For the library, we choose to implement MEKF.

On the other hand, the Unscented Kalman Filter method was proposed by Julier and Uhlmann \cite{julier_new_1995} in 1995 . It belongs to a class of filters called Sigma-Point Kalman Filters. Instead of using the Taylor series expension, UKF use an unscented transformation \cite{uhlmann1995dynamic} to approximate the nonlinear projection of mean and covariance of a probability distribution. For the library, we choose to implement the filter proposed by Crassidis et al. called UnScented QUaternion Estimator (USQUE) \cite{crassidis_unscented_2003}, based on UKF. They used the same representation as MEKF.

\subsubsection{Nonlinear Observer}

The theory of the state observer was first introduced by Kalman and Bucy for a linear system in a stochastic environment. Then Luenberger \cite{david1971introduction} made a general theory of observers for deterministic linear systems, introducing the notions of observer and reduced minimum observer \cite{primbs1996survey}. Mahony et al. \cite{mahony_nonlinear_2008} have proposed an nonlinear observer called Constant Gain Observer (CGO), termed the explicit complementary filter, that provides attitude estimates as well as gyro bias estimates. 

One problem of Kalman Filters is to find the initial values in the covariance (matrix $Q$) because it is difficult to relate it to real physical quantities. The covariance matrix is not involve in nonlinear observer theory. This is one of the reasons there has been an increasing interest for nonlinear observer design the last decades. Another reason is their low computational need.

\subsubsection{Methods Based on Solutions of the Wahba's Problem}

The attitude determination problem for vector observation was first formulated as a least squares estimation problem by Wahba \cite{wahba_least_1965} in 1965 :\\


Given the two sets of $n$ vectors $\{ \textbf{r}_{1},...,\textbf{r}_{n} \}$ and $\{ \textbf{b}_{1},...,\textbf{b}_{n} \}$, $n \geqslant 2 $, where each pair $(\textbf{r}_{i},\textbf{b}_{i})$ corresponds to a generalised vector, $\textbf{x}_{i}$, find the proper orthogonal matrix, $A$, which bring the first into the best least squares coincidence with the second. That is, find $A \in SO(3)$ which minimises the function $J$ defined as:

\begin{equation}
J(A) = \frac{1}{2} \sum_{i=1}^n a_i||b_i-Ar_i||^2
\label{Wahba_loss_function}
\end{equation}

The equation (\ref{Wahba_loss_function}) can be written with a quaternion representation which gives:

\begin{equation}
J(\mathbi{q}) = \frac{1}{2} \sum_{i=1}^n a_i||b_i-R(\mathbi{q})r_i||^2
\label{QUEST}
\end{equation}

The matrix $R(\mathbi{q})$, is the rotation matrix that corresponds to the quaternion $\mathbi{q}$ (equation \ref{quat_to_rot}). The QUEST algorithm are conducted in the following manner. Let

\begin{equation}
B = \sum_{i=1}^N a_i\textbf{b}_i \textbf{r}_i^T
\end{equation}

Then, we can rewrite the equation (\ref{QUEST}) as follow

\begin{equation}
J(\mathbi{q}) = \frac{1}{2} \sum_{i=1}^n a_i  - tr(R(\mathbi{q})B^T)
\label{QUEST_bis}
\end{equation}

Due to the fact that the representation of the attitude matrix is a homogenous quadratic function of $\mathbi{q}$, we can say that

\begin{equation}
tr(R(\mathbi{q})B^T) = \mathbi{q}^TK\mathbi{q}
\label{QUEST_ter}
\end{equation}

where $K$ is the symmetric traceless matrix such as

\begin{equation}
K = \begin{pmatrix} S-tr(B)I_3 & \textbf{z} \\ \textbf{z}^T & tr(B)
\end{pmatrix}
\end{equation}

with

\begin{equation}
\left\{\begin{array}{l}
S = B + B^T\\
\textbf{z} = \sum_ia_i\textbf{b}_i\times\textbf{r}_i
 \end{array}
\right.
\end{equation}


The minimisation problem from Equation (\ref{QUEST_bis}) is now a maximisation problem. thus, the solution $\mathbi{q}_{opt}$ is the eingen vector correspond to the largest eigen value of $K$ \cite{markley1999estimate}. The main problem of QUEST is that it does not take into account of all the previous observations performed since the beginning. A recursive QUEST algorithm (REQUEST) proposed by Bar-Itzhack manages this problem.  Contrary to other methods, the QUEST algorithms do not estimate the gyroscope bias. Which can make it less robust in some cases.

There are also other methods based on Wahba's problem like the Davenport's q-method \cite{weighted1971nasa} proposed in 1971 considered more robust than QUEST. Or the method ESOQ proposed by Mortari \cite{mortari1997esoq}  in 1997. However, for this library, we limit our investigation to REQUEST.\\



\subsubsection{Particle Filtering Method}

Particle filtering method is based on Monte-Carlo method\cite{metropolis1949monte}. The main idea is to use a large number of samples called particles to estimate a nonlinear function \cite{chen_bayesian_2003} and \cite{terejanu2009tutorial}. The set of $N$ particle with associated weight at time $k$ are denoted by $\{\textbf{x}^{(i)}_{k},w^{(i)}_{k}\}$. Initially, the particle are drawn from a proposal distribution. The PF is divided into three step, prediction, update, resampling, that constitute a filter cycle:

\begin{itemize}
\item Prediction
\end{itemize}

The first step is to propagate all particles $\{\textbf{x}^{(i)}_{k},w^{(i)}_{k}\}$ to $\{\textbf{x}^{(i)}_{k+1},w^{(i)}_{k+1}\}$ with the nonlinear function $f$ while the weight remain unchanged : 

\begin{equation}
\textbf{x}^{(i)}_{k+1} = f(\textbf{x}^{(i)}_{k},\textbf{u}_{k}) + \textbf{w}_{k}^{(i)}\\
\end{equation}

where $N$ sample $\textbf{w}_{k}^{(i)}$ of the process noise (supposed gaussian in our case) are drawn.

\begin{itemize}
\item Update/Correction
\end{itemize}

At the update step, the weights associated with each particle is updated based on the likelihood function $p( \textbf{z}_{k+1}| \textbf{x}_{k+1}^{(i)})$.
\begin{equation}
\left\{ 
\begin{array}{l}
w_{k+1}^{(i)} = w_{k}^{(i)} p( \textbf{z}_{k+1}| \textbf{x}_{k+1}^{(i)})\\
\tilde{w}_{k+1}^{(i)} = \frac{w_{k+1}^{(i)}}{\sum_iw_{k+1}^{(i)}}
\end{array}
\right.
\end{equation}

The weight are normalised such as we have $\sum_i\tilde{w}_{k+1}^{(i)}=1$. To finish, the mean is computed with the following equation:

\begin{equation}
 \textbf{x}_{k+1}^+ = \sum_{i=1}^N\tilde{w}_{k+1}^{(i)} \textbf{x}_{k+1}^{(i)}
 \end{equation}



\begin{itemize}
\item Resampling/Regularisation
\end{itemize}

A common problem with PF is the degeneracy phenomenon. After some iterations, more and more particles will have a weak weight, which make them negligible and useless.  To overcome this problem, a regularisation step called resampling is processed. It is not needed to perform PF but it allows to maintain a good performance relative to degeneracy problem. The idea of resampling is to eliminate particles having feeble weight and focus on strong weight particles. In order to measure degeneracy, the number of effective samples $N_{eff}$ is computed (equation \ref{N_eff}). If  $N_{eff}$ is lower than a threshold, the resampling step is processed.

\begin{equation}
 N_{eff} = \sum_{i=1}^N\frac{1}{(\tilde{w}_k^{(i)})^2}
 \label{N_eff}
 \end{equation}


The choice of the threshold is important. Indeed, a high threshold gives $N_{eff} \approx N$, but that does not mean necessarily we have a lot of efficient particles. On the contrary, it means that most of the particle are identical making large computations unnecessary. Moreover, with a low threshold, many particles are neglected making computations less efficient. In general, the threshold is fixed at  $\frac{N}{2}$ or $\frac{N}{3}$. There are several types of resampling, a comparison is made in \cite{douc2005comparison}. We took the resampling algorithm described in \cite{arulampalam2002tutorial} because it is the most commun. Cheng, Yang and Crassidis have propose an attitute estimation with a particle filter method \cite{cheng_particle_2010}. We have also considered the revisited algorithm of Cheng \cite{chang_particle_2014} based on bootstrap filter\cite{gordon1993novel}. We have chosen to implement this one because they use the same approach as USQUE \cite{crassidis_unscented_2003} and MEKF\cite{markley2003attitude}. We decided to explore the performance of this three algorithms with the same model. The particle filter provides an estimate of the nonlinear system without assuming that the noise is Gaussian. So it is more robust to any situation, which makes it more performant in the context of a real applications. But an important computation time is required and that is a limitation for a real-time application.



\section{Performance Evaluation}

\subsection{Score definition}
We have access through our designed ground truth (simulated or from computer vision technics) to the true attitude $\mathbi{q}^{real}_k$ at time $k$ of the reference frame $\{c\}$ related to $\{a\}$.
By computation, each method returns an estimate attitude $\mathbi{q}^{est}_k$ at time $k$ (see figure \ref{test_method}). 

\begin{equation}
 \delta \mathbi{q}_k = \mathbi{q}^{real}_k \otimes (\mathbi{q}^{est}_k)^{-1} =  \begin{pmatrix}  \delta \textbf{q}_{k} \\ \delta q_{w_k}  \end{pmatrix}
\end{equation}

The error-MRP vector related to $ \delta \mathbi{q}_k$ is given by

\begin{equation}
 \delta \textbf{p}_k  = 4[\frac{\delta \textbf{q}_{k}  }{(1+\delta q_{w_k})}]
\end{equation}


Thus, we can calculate the attitude error given by the root mean square (RMS) of $\delta \textbf{p}_k$:

\begin{equation}
\epsilon_k  = RMS (\delta \textbf{p}_k) = \sqrt{\frac{1}{3}( \delta p_{k_x}^2 +\delta p_{k_y}^2+\delta p_{k_z}^2)  }
\label{error_definition}
\end{equation}

The mean of the error is given by :

\begin{equation}
\bar{\epsilon}  = \frac{1}{n}\sum_{i=1}^n \epsilon_i \\
\label{mean_var_error_def}
\end{equation}

With $n$ the number of data at our disposal. We use this information (mean error) for our performance study. 

\begin{figure}[!h]
\centering
\includegraphics[scale=0.40]{images/test_method.png}
\caption{Algorithm designed to compute attitude estimation methods' performances}
\label{test_method}
\end{figure}


To differentiate the methods, we assigned  some ''scores'' in order to highlight the behaviour of these methods in different configurations. We defined these ''scores'' as:

\begin{align}
s_1 = \bar{\epsilon}\\
s_2 = \frac{1}{\bar{\epsilon} \times \bar{\tau}}
\label{score}
\end{align}

With $\bar{\epsilon}$ average attitude error (voir \ref{error_definition}) et $\bar{\tau}$ the average computation time. \\

\subsection{Simulated environment}

We  modelled our inertial sensors based on the Fossen's model \cite{fossen_handbook_2011}. The noise generated here is Gaussian. In this study, sensor bias are not considerate as constant. We proceed 30 consecutive simulations in order to get valid results. We've used  specification described in table \ref{spec_imu}. 

\begin{table}[!h]
\begin{tabular}{|c|c|c|}
\hline
Sensor & Bias instability  & Measurement noise\rule[-2pt]{0pt}{10pt} \\
\hline
\hline
 Gyroscope & $0.316\times 10^{-4} $  $\mu$rad.s$^{-\frac{3}{2}}$ & $ 0.316$  $\mu$rad.s$^{-\frac{1}{2}}$ \rule[-1.5pt]{0pt}{13pt}\\
 \hline
Magnetometer & $0.0023$ mGauss$/\sqrt{Hz}$ & 0.05 mGauss  \rule[-1.5pt]{0pt}{13pt}\\
 \hline
Accelerometer &  $0.002$ m.s$^{-2}/\sqrt{Hz} $ & 0.02 m.s$^{-2}$ \rule[-1.5pt]{0pt}{13pt}\\
 \hline
\end{tabular}
\caption{Calibrated IMU data performance specification, Gyrometer, Accelerometer, Magnetometer}
\label{spec_imu}
\end{table}



All simulations started with an initial attitude error $\delta\mathbi{q}_0$ randomly generated through normal distribution $\mathcal{N}(20,50^2)$. The results of score $s_1$ (resp $s_2$) are shown in figure \ref{histo_s1_add} (resp \ref{histo_s2_add}). Mean computation times of all methods required for one iteration is given in table (\ref{mean_time}).

\begin{table}[!h]
\centering
\begin{tabular}{|c|c|}
\hline
Method & Mean Computation time \\
\hline
\hline
CGO & $7.88 \times 10^{-6} s$\rule[-2pt]{0pt}{10pt}\\
 \hline
MEKF & $1.12\times 10^{-4} s$\rule[-2pt]{0pt}{10pt}\\
 \hline
 USQUE & $1.63\times 10^{-4} s $\rule[-2pt]{0pt}{10pt}\\
 \hline
PF & $4.5\times 10^{-2} s$\rule[-2pt]{0pt}{10pt}\\
 \hline
REQUEST& $1.91\times 10^{-4} s$ \rule[-2pt]{0pt}{10pt} \\
 \hline
\end{tabular}
\caption{Mean computation times for one iteration}
\label{mean_time}
\end{table}



\begin{figure}[!h]
\begin{subfigure}{.5\textwidth}
\centering
\includegraphics[scale=0.28]{images/histo_s1_add.png}
\caption{Scores $s_1$}
\label{histo_s1_add}
\end{subfigure}

\begin{subfigure}{.5\textwidth}
\centering
\includegraphics[scale=0.28]{images/histo_s2_add.png}
\caption{Scores $s_2$ }
\label{histo_s2_add}
\end{subfigure}
\caption{Scores $s_1$ (accuracy) and $s_2$ (accuracy and computation time ratio) of methods where noise is AWGN with 3 strength (weak, moderate, strong)}
\end{figure}

\subsection{Experimental validation}

We have designed a system that can be easily repeatable in laboratory. We used a smartphone as our IMU and a calibrated webcam (see figure \ref{situation_validation} ). With this system, we have defined our ground-truth as follows. A marker is stuck on the smartphone and we calculate an estimate of the attitude of the marker (which represents the smartphone) relative to the reference frame of the webcam $\{w\}$. We used the library \texttt{ArUco}  as optical feedback. At the same time, we get the data from the IMU of the smartphone to proceed the attitude estimation with our different algorithms.


\begin{figure}
\centering
\includegraphics[scale=0.35]{images/situation_validation.png}
\caption{Representation of different reference frame used in this experiment. Here, only $ \text{\mathbi{q}}_w^a $  does not change.}
\label{situation_validation}
\end{figure}



At time $t = 0$, the camera and the smartphone has been set such as attitude quaternions $\mathbi{q}^a_w$, $\mathbi{q}^c_a$ et $\mathbi{q}^w_c$ are approximatively known. The camera being fixed, $\mathbi{q}^a_w $ does not vary. Moreover, from figure \ref{situation_validation}  we have:

\begin{equation}
\mathbi{q}^c_a =( \mathbi{q}^c_w \otimes \mathbi{q}^a_w )^{-1}
\end{equation}

\begin{figure}
\centering
\includegraphics[scale=0.35]{images/output_aruco.png}
\caption{exemple of output of \texttt{ArUco} with 6 markers. Here the object-fixed reference frame $\{c\}$ is drawn}
\label{situation_validation}
\end{figure}

All methods give an estimate of $\mathbi{q}^c_a$ , that are compared with the data from the camera. The optical feedback provides a ground truth estimate for this experiment. To estimate the error that can occur with the optical feedback. We were careful in the experimentation to avoid the occlusion of the marker that is required to estimate the orientation. During this experiment, we applied a rotation of 90 degree around the $z_c$  on the smartphone in clockwise direction. Then, another rotation of 90 degree around the $z_c$ on the smartphone but in anticlockwise direction to return to the initial state. In order to study possible divergence of all methods. 




%%%%%%%%%%%%
% Conclusion - Discussion
%%%%%%%%%%%%

\section{Conclusion}

notre library est trop bien...


\begin{backmatter}

\section*{Competing interests}
  The authors declare that they have no competing interests.

\section*{Author's contributions}
    Text for this section \ldots

\section*{Acknowledgements}
This work was supported by the ARAP program of ASTAR and by the university Sorbonne UPMC.

\bibliographystyle{bmc-mathphys} 
\bibliography{bmc_article}      

\end{backmatter}





\end{document}