\section{Conclusion}

In this paper, we presented our open source library \texttt{OpenMotion} and its different functionalities. Our aim is to make it a powerful tool for developing application or new algorithms. Combining this library with our comparative framework \cite{braudcomparison} could be an interesting tool for research. A performance study has been processed too in order to show the capacities of the algorithms present in the library. Thus, the user has access to the algorithm that fit the best to his needs and the open source feature allows him to have a better understanding of the algorithms. For future work, we continue to develop the library and focus on his robustness and adaptability to every type of smartphone. With the developing sensor calibration, better accuracy is expected too. 


\begin{backmatter}

\section*{Competing interests}
  The authors declare that they have no competing interests.

\section*{Author's contributions}
    Text for this section \ldots

\section*{Acknowledgements}
This work was supported by the ARAP program of ASTAR and by the university Sorbonne UPMC.

\bibliographystyle{bmc-mathphys} 
\bibliography{bmc_article}      

\end{backmatter}
