
\begin{abstractbox}

\begin{abstract} % abstract

The abstract goes here
\end{abstract}

\begin{keyword}
\kwd{IMU}
\kwd{Sensor Fusion}
\kwd{Sensor Calibration}
\end{keyword}

\end{abstractbox}


\end{frontmatter}

\section{Introduction}


%context + motivation

Since the last decades, it is easily observable that smartphones market has grown drastically. Indeed, this devices propose convenient functionalities for household. On the other hands, the different kind of sensor available and easily accessible make smartphones an interesting tool for research.  Nowadays, some applications where smartphone can be used attract the attention of researchers such as Indoor Navigation\cite{huang2009survey}. Severals solution for this has been proposed with different approaches like using only computer vision technics \cite{desouza2002vision,bonin2008visual} or wireless networks\cite{gu2009survey}. Another approaches is to sense the motion by using some inertial sensors. Among its sensor and according to the smartphone, an Inertial Measure Unit (IMU) is vacant and used when coupling with a magnetometer as an Attitude and Heading Reference System (AHRS). However, an AHRS sensor may be not robust or accurate in certain situation due to unpredictable event or sensor deficiencies. Moreover, for Indoor Localization or Navigation applications, it is crucial to have an algorithm that propose an optimal estimation of the attitude in term of accuracy and computation time requirement. This is the reason why researchers are interested in sensor fusion approaches because it overcomes some problems such as sensor drifting. 

Since 1960, severals data fusion has been developed after a seminal framework on data fusion has been initiated by Kalman and  Bucy \cite{kalman_new_1960}. In this paper, the most common family of data fusion algorithm is presented briefly. However, implementing this kind of algorithm require time and resources. Furthermore, the available software of data fusion remains general and may be not adaptable for the inertial case. For this reason, we propose an open source and free library called \texttt{OpenMotion} that contains different family of sensor fusion algorithms designed to fit to the inertial case. For this paper, the different functionalities proposed by the library are presented. In a simulated environment, a performance study of all methods has been processed and the values are shown. To quantify the performances, we determined 2 kind of scores,  $s_1$ representing the accuracy of the method and $s_2$ that takes into account the accuracy and the computation time. The $s_2$ give better visibility of the methods' effectiveness for usage in embedded devices, which is interesting for smartphone application. Furthermore, working only in a simulated environment is not enough to have a good visibility. Moreover, in our context, it is very difficult to obtain a reliable ground truth. For this study, we designed the ground truth  based on computer vision methods. 