\section{Calibration and initialization}

\subsection{Probl\'ematique de la calibration}

	Les syst\`emes embarqu\'e comme les smartphones disposent pour la plupart de capteurs servant au syst\`eme ou \`a l'utilisateur pour se renseigner sur des grandeurs utiles quant \`a l'appareil lui-m\^eme ou bien \`a son environnement. Ces capteurs sont donc vari\'es mais on peut notamment distinguer les capteurs constituant une centrale inertielle ou IMU pour Inertial Measurement Unit que sont les acc\'el\'erom\`etres et les gyroscopes 3 axes. Ces capteurs sont de plus en plus associ\'es avec un magn\'etom\`etre. Ainsi, ces capteurs peuvent constituer un AHRS pour Attitude and Heading Reference System ayant pour but de rep\'erer l'appareil dans l'espace 3D. L'acc\'el\'erom\`etre sert ici \`a mesurer l'acc\'el\'eration lin\'eaire, le gyroscope la vitesse angulaire et le magn\'etom\`etre le champ magn\'etique terrestre local et sert dans la plupart des cas de compas. Dans le cas d'un smartphone, ces capteurs sont low-cost et de type MEMS pour Micro-Electro-Mechanical Systems. D\`es lors, le signal utile \'etant faible, l'erreur induite peut-\^etre du m\^eme ordre de grandeur et donc conduire \`a des impr\'ecisions importantes, notamment dans le cas des AHRS qui font l'objet d'une fusion de donn\'ees. Il convient donc d'\'eliminer ou tout du moins de r\'eduire ces erreurs via une calibration initiale de l'ensemble des capteurs utilis\'es.

\subsection{Types et mod\`ele d'erreur}

	Tous les types de capteurs pr\'esentent des biais, facteurs d'\'echelle (scale factors), des erreurs d'inter-couplage (misalignements et cross axis sensitivities) et dans une certaine mesure du bruit al\'eatoire. Des erreurs d'ordre sup\'erieur, des erreurs d'inter-couplage d'acc\'el\'eration angulaire ou encore des erreurs sp\'ecifiques comme les softs et irons distorsions pour le magn\'etom\`etre peuvent \'egalement se produire.
\\
\\
Pour l'ensemble des capteurs, un mod\`ele rendant compte des erreurs lors des diff\'erentes mesures peut-\^etre propos\'e :

\begin{displaymath}

	X_i = \textbf{M} \textbf{S} \cdot \left( X_{i,raw} - b \right) + n_i
	
\end{displaymath}

o\`u $X_i$ est la mesure du capteur comprenant les erreurs, $X_{i, raw}$ la mesure r\'eelle du capteur,  $b$ le biais,  $\textbf{S}$ la matrice rendant compte des scales factor errors, $\textbf{M}$ la matrice des misalignement entre les axes du capteur et $n_i$ le bruit al\'eatoire avec $i \in \{x,y,z\}$.
\\
\\
Les diff\'erents matrices sont alors 

\begin{displaymath}

	\textbf{S} = 
	
	\begin{pmatrix}
	
		s_{xx} & s_{xy} & s_{xz} \\
		
		s_{yx} & s_{yy} & s_{yz} \\
		
		s_{zx} & s_{zy} & s_{zz} 
		
	\end{pmatrix} 
	
	\text{ et } \textbf{M} = 
	
	\begin{pmatrix}
	
	   1 & -\alpha_{yz} & \alpha_{zy} \\
	
	   0 & 1 & -\alpha_{zx} \\
	
	   0 & 0 & 1 
	
	\end{pmatrix} 
	
\end{displaymath} 

o\`u $\alpha_{ij}$ avec $i,j \in \{x,y,z\}$ est l'angle de misalignement entre l'axe \emph{i} du rep\`ere du capteur avec celui de la plateforme selon l'axe \emph{j} de la plateforme.
\\
\\
Enfin le biais est repr\'esent\'e par un vecteur 
\begin{displaymath} 

	b = 
	
	\begin{pmatrix}
	
		b_x\\
		
		b_y \\
		
		b_z
		
	\end{pmatrix}
	
\end{displaymath} 

\subsection{M\'ethodes adopt\'ees}

	Une premi\`ere calibration d'usine est effectu\'ee mais s'av\`ere souvent impr\'ecise ou obsol`ete du fait de l'usure des capteurs. D\`es lors, une nouvelle calibration s'av\`ere n\'ecessaire.
\\
\\
	La calibration des capteurs se fera dans un premier temps ind\'ependamment les unes des autres et de mani\`ere statique \emph{i. e.} \`a l'initialisation. Il est \a noter que nous ne prendrons pas en compte ici l'influence de la temp\erature puisqu'il est d\'ej\`a trait\'e dans la calibration d'usine.

\subsubsection{Acc\'el\'erom\`etre}

	Freescale \copyright ([Freescale]) propose une m\'ethode de calibration lin\'eaire de l'acc\'el\'erom\`etre 3 axes.
\\
\\
	Elle repose sur le fait que la somme des amplitudes des composantes normalis\'ees du champ gravitationnel terrestre sur les 3 axes du capteur doit \^etre \'egale \`a l'amplitude totale de celui-ci quand le capteur est dans un \'etat quasi statique. Elle consiste \`a placer l'appareil dans 8 positions donn\'ees pour chaque axe puis d'utiliser une m\'ethode d'optimisation de moindres carr\'es pour d\'eterminer les 12 param\`etres de calibration que sont les scale factors et le biais.
\\
\\
Cette m\'ethode ne permet toutefois pas de d\'eterminer les misalignements. 

\subsubsection{Gyroscope}

	\ref{Gyroscope} propose une m\'ethode de calibration du gyroscope. Elle repose sur le fait que la somme des mesures selon les 3 axes du gyroscope doit \^etre \'egale \`a la vitesse angulaire en entr\'ee. 
\\
\\
	2 \'etapes sont alors n\'ecessaires \`a la calibration du gyroscope. La premi\`ere consiste en des rotations horaires et anti-horaires permettant d'\'eliminer la vitesse angulaire de rotation de la terre et le biais du gyroscope en prenant le carr\'e de la norme de la diff\'erence des int\'egrales des mesures dans les 2 sens puis en utilisant une m\'ethode de moindre carr\'e. Les scales factors et les angles de misalignements sont ainsi d\'etermin\'es. Ensuite, les angles de misalignements et les scales factors sont utilis\'es pour estimer le biais via la diff\'erence des carr\'es de l'\'equation de la somme des vitesses angulaires pour 2 positions diff\'erentes.

\subsubsection{Magn\'etom\`etre}

	Vasconcelos et al. \ref{MLE} proposent un algorithme dit de maximum likelihood estimation (MLE) qui utilise l'estimation issue de "Two-Step" (\ref{Two-Step}), \emph{i. e.} une m\'ethode d'estimation de moindre carr\'e suivie d'un filtrage \'etendu de Kalman (EKF), comme approximation initiale et qui traite le bruit de mesure comme une distribution normale.
\\
\\
	Le probl\`eme de minimisation  la fonction  de vraisemblance (likelihood) qui est quadratique revient alors \`a d\'eterminer les points de l'ellipso\"{i}de  qui correspondent au mieux aux mesures du capteur.
\\
\\
	Ce probl\`eme est r\'esolu via la m\'ethode du gradient et une m\'ethode de Newton descendante pour des espaces euclidiens \ref{Newton} et la r\`egle de Armijo pour la d\'etermination de la taille du pas. Le gradient et la matrice hessienne de la fonction de log-likelihood sont eux calcul\'es analytiquement. 

\subsubsection{Calibration on-board}

	Une calibration on-board peut \'egalement s'av\'erer n\'ecessaire du fait d'un besoin de pr\'ecision accrue ou de l'obsolescence de la calibration d'usine due \`a l'usure des capteurs.
